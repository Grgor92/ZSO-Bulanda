\begin{itemize}
\item Ustalanie zasad dostępu do systemu (zasada najmniejszych uprawnień – Least Privilege).
    \item Dostęp do zasobów tylko z sieci wewnętrznej. \newline
    Nowy użytkownik dostaje minimalne uprawnienia, które można poszerzyć w przypadku potrzeby które wynikną w trakcie korzystania z systemów. 
    
    \item Definiowanie ról i grup użytkowników z odpowiednimi uprawnieniami.
    \newline Użytkownik zostaje dołączony do grupy użytkowników ze względu na zajmowane stanowisko. Każda grupa użytkowników dziedziczy uprawnienia po wyższej, lecz są one ograniczone w zależności od zajmowanego stanowiska: 
    
    \begin{itemize}
        \item{Pracownicy BOI.}
        \begin{itemize}
            \item {Dostęp do danych spraw na poziomie odczytu}
        \end{itemize}
        
        \item{Sędziowie}
        \begin{itemize}
            \item {dostęp do systemu repertoryjno-biurowy na poziomie odczytu danych sprawy}
            \item {możliwość dodawania oraz edycji orzeczeń dot. spraw}
        \end{itemize}
  
        \item{Protokolanci}
        \begin{itemize}
            \item {Dodawanie oraz edycja przebiegu i wyników rozpraw}
            \item {Wprowadzenie oraz edycja terminów rozpraw}
            \item {Wysyłanie korespondencji do stron sprawy}
        \end{itemize}
        
        \item{Pracownicy Sekretariatu}
        \begin{itemize}
            \item {Rejestracja spraw oraz ich szczegółów}
            \item {Edycja danych sprawy w zakresie terminów rozpraw,ich wyników oraz wniosków wpływających w sprawach}
            \item {Zarządzanie korespondencją (wysłana, odebrana, wysyłanie akt do biegłych}
            \item {Dostęp do wykazów i kontrolek prowadzonych w wydziale}
        \end{itemize}
        
        \item{Kierownicy Sekretariatu}
        \begin{itemize}
            \item {Tworzenie, edycja oraz usuwanie użytkowników}
            \item {Tworzenie grup użytkowników oraz nadawanie uprawnień w systemie repertoryjno-biurowym}
            \item {Uprawnienia do wykazów statystycznych}
        \end{itemize}
    \end{itemize}
    .
     \newline Osobną grupą użytkowników jest Biuro podawcze.
    \begin{itemize}
        \item {Biuro podawcze}
        \begin{itemize}
            \item {Wysyłanie i odbieranie korespondencji}
        \end{itemize}
    \end{itemize}

    \item Regularne przeglądy uprawnień użytkowników i ich aktualizacja.
\end{itemize}