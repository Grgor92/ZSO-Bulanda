\documentclass[12pt,a4paper]{article}
\usepackage[polish]{babel}
\usepackage[T1]{fontenc}
\usepackage[utf8x]{inputenc}
\usepackage{hyperref}
\usepackage{url}
\usepackage{graphicx}

\addtolength{\hoffset}{-1.5cm}
\addtolength{\marginparwidth}{-1.5cm}
\addtolength{\textwidth}{3cm}
\addtolength{\voffset}{-1cm}
\addtolength{\textheight}{2.5cm}
\setlength{\topmargin}{0cm}
\setlength{\headheight}{0cm}

\begin{document}

\title{Dokumentacja. \\ Polityka bezpieczeństwa sądu}

\author{Nika Lytvynchuck, Łyszczarz Klaudia, Bobak Grzegorz}
\date{\today}

\maketitle

\begin{abstract}
\centering{Zarys polityki bezpieczeństwa systemów informatycznych w sądzie}
\end{abstract}

\newpage

\tableofcontents
\listoftables
\listoffigures

\newpage

\section{Tytuł}
\section{Cel}
Celem projektu jest opracowanie kompleksowej dokumentacji obejmującej politykę bezpieczeństwa systemów informatycznych w sądzie, aby zapewnić ich niezawodność oraz bezpieczeństwo.

\section{Zakres}
\subsection{Analiza wymagań}
Polityka bezpieczeństwa w sądzie powinna obejmować właściwe zabezpieczenia w zakresie ochrony danych, niepowołanego dostępu, zarządzania i monitorowania ruchu sieciowego, ochrony systemów informatycznych, określać postępowanie w przypadku ataków sieciowych, wystąpienia incydentów oraz zapełniać odpowiednią świadomość i edukację użytkowników.
\subsection{Wymagania funkcjonalne i niefunkcjonalne}
\input{wymagania}
\subsection{Dobór technologii}
\input{technologie}

\section{Polityka bezpieczeństwa}
\subsection{Zarządzanie dostępem}
\begin{itemize}
\item Ustalanie zasad dostępu do systemu (zasada najmniejszych uprawnień – Least Privilege).
    \item Dostęp do zasobów tylko z sieci wewnętrznej. \newline
    Nowy użytkownik dostaje minimalne uprawnienia, które można poszerzyć w przypadku potrzeby które wynikną w trakcie korzystania z systemów. 
    
    \item Definiowanie ról i grup użytkowników z odpowiednimi uprawnieniami.
    \newline Użytkownik zostaje dołączony do grupy użytkowników ze względu na zajmowane stanowisko. Każda grupa użytkowników dziedziczy uprawnienia po wyższej, lecz są one ograniczone w zależności od zajmowanego stanowiska: 
    
    \begin{itemize}
        \item{Pracownicy BOI.}
        \begin{itemize}
            \item {Dostęp do danych spraw na poziomie odczytu}
        \end{itemize}
        
        \item{Sędziowie}
        \begin{itemize}
            \item {dostęp do systemu repertoryjno-biurowy na poziomie odczytu danych sprawy}
            \item {możliwość dodawania oraz edycji orzeczeń dot. spraw}
        \end{itemize}
  
        \item{Protokolanci}
        \begin{itemize}
            \item {Dodawanie oraz edycja przebiegu i wyników rozpraw}
            \item {Wprowadzenie oraz edycja terminów rozpraw}
            \item {Wysyłanie korespondencji do stron sprawy}
        \end{itemize}
        
        \item{Pracownicy Sekretariatu}
        \begin{itemize}
            \item {Rejestracja spraw oraz ich szczegółów}
            \item {Edycja danych sprawy w zakresie terminów rozpraw,ich wyników oraz wniosków wpływających w sprawach}
            \item {Zarządzanie korespondencją (wysłana, odebrana, wysyłanie akt do biegłych}
            \item {Dostęp do wykazów i kontrolek prowadzonych w wydziale}
        \end{itemize}
        
        \item{Kierownicy Sekretariatu}
        \begin{itemize}
            \item {Tworzenie, edycja oraz usuwanie użytkowników}
            \item {Tworzenie grup użytkowników oraz nadawanie uprawnień w systemie repertoryjno-biurowym}
            \item {Uprawnienia do wykazów statystycznych}
        \end{itemize}
    \end{itemize}
    .
     \newline Osobną grupą użytkowników jest Biuro podawcze.
    \begin{itemize}
        \item {Biuro podawcze}
        \begin{itemize}
            \item {Wysyłanie i odbieranie korespondencji}
        \end{itemize}
    \end{itemize}

    \item Regularne przeglądy uprawnień użytkowników i ich aktualizacja.
\end{itemize}
\subsection{Zarządzanie hasłami}
\begin{itemize}
    \item Wymuszanie silnych haseł (długość, złożoność, unikalność).
    \begin{itemize}
        \item {Hasło musi zawierać jeden znak specjalny, znak numeryczny, dużą literę, długość min. 14 znaków}
        \item {Sprawdzanie haseł pod kątem najczęściej stosowanych}
        \item {Hasło nie może być powiązane z właścicielem konta lub loginem, m. in. Nazwa użytkownika, data urodzenia itp.}
        \item {Hasła muszą być szyfrowane}
    \end{itemize}
    
    \item Okresowa zmiana haseł.
    \begin{itemize}
        \item{Wymuszanie zmiany hasła co 3 miesiące}
    \end{itemize}
    
    \item Blokowanie kont po kilku nieudanych próbach logowania.
    \begin{itemize}
        \item{Blokowanie konta po 3 nieudanych próbach}
        \item{Możliwość odblokowania konta posiada jedynie administrator sieci}
    \end{itemize}
    
    \item Korzystanie z menedżerów haseł i polityki zero trust.
    \begin{itemize}
        \item {Administrator sieci posiada zaufany menadżer haseł .......... do zarządzana systemów, użytkownikami oraz siecią}
    \end{itemize}
\end{itemize}
\subsection{Aktualizacje i zarządzania poprawkami}
\begin{itemize}
    \item Regularne aktualizacje systemu operacyjnego i oprogramowania.
    \begin{itemize}
        \item {Aktualizacje podstawowych aplikacji są wykonywane za pomocą narzędzia Endpoint Manage Central}
        \item {Systemy dostarczane przez zewnętrznych dostawców (nagrania z rozpraw, system ksiąg wieczystych, systemów repertoryjno-biurowych) są aktualizowane przez firmy zewnętrzne}
        \item {Systemy pocztowe są aktualizowane przez Administratora sieci przy asyście firm zewnętrznych}
        \item {Systemy operacyjne są wymuszane przez Administratora sieci}
    \end{itemize}
    
    \item Automatyczne instalowanie krytycznych poprawek zabezpieczeń.
    \begin{itemize}
        \item {Zastosowanie polityki zero trust w przypadku ruchu sieciowego}
    \end{itemize}
    
    \item Testowanie poprawek przed wdrożeniem na środowisko produkcyjne.
    \begin{itemize}
        \item {}
        \item {}
    \end{itemize}
    
    \item Eliminowanie zbędnego i przestarzałego oprogramowania. 
    \begin{itemize}
        \item {...}
        \item {Oprogramowanie skanujące sieć oraz stacji pod względem przestarzałych oprogramowań - Tanable Nessus}
    \end{itemize}
    
\end{itemize}
\subsection{Monitorowanie i wykrywanie zagrożeń}
Do analizu ruchu sieciowego, logów systemowych i sieciowych wykorzystujemy narzędzie EndpointProtector (OPISAC SZCZEGÓLOWO JAK DZIAŁA)
\begin{itemize}
    \item Wdrożenie systemów IDS/IPS.
    \item Korzystanie z systemów EDR/XDR do monitorowania aktywności.
    \item Analiza logów systemowych i sieciowych.
    \item Wykrywanie anomalii i nieautoryzowanych działań.
\end{itemize}
\subsection{Ochrona danych}
\begin{itemize}
    \item Szyfrowanie danych przechowywanych i przesyłanych.
    (Dyski zabezpieczane bitlockerem) 
    \item Implementacja zasad tworzenia kopii zapasowych.
    \item Zabezpieczenie dostępu do nośników danych.
    \item Bezpieczne usuwanie danych.
\end{itemize}
\subsection{Zarządzanie urządzeniami i stacjami roboczymi}
\begin{itemize}
    \item Polityka instalacji oprogramowania.
    \begin{itemize}
        \item {Sprawdzanie aktualizacji na testowym stanowisku}
        \item {Sprawdzanie aplikacji po aktualizacji na testowym stanowisku}
    \end{itemize}
    
    \item Kontrola dostępu do urządzeń USB i nośników danych.
    \item Konfiguracja zapór ogniowych.
    \item Wdrożenie zasad ograniczonego uruchamiania aplikacji.
\end{itemize}
\subsection{Edukacja i świadomość użytkowników}
\begin{itemize}
    \item Regularne szkolenia z zakresu cyberbezpieczeństwa.
    \item Symulacje ataków phishingowych.
    \item Informowanie o zagrożeniach i procedurach bezpieczeństwa.
\end{itemize}
\subsection{Polityka reagowania na incydenty}
\begin{itemize}
    \item Opracowanie planu reagowania na incydenty.
    \item Wyznaczenie zespołu SOC.
    \item Procedury zgłaszania incydentów.
    \item Analiza i dokumentowanie incydentów.
\end{itemize}
\subsection{Bezpieczne korzystanie z sieci}
\begin{itemize}
    \item Wymuszanie VPN dla zdalnych połączeń. (Praca zdalna tylko na urządzeniu sądowym, które przeszło audyt IT, praca możliwa tylko po połączeniu się odpowiednio skonfigurowanym VPN do sieci wewnętrznej sądu)
    \item Separacja sieci dla krytycznych systemów.
    \item Blokowanie dostępu do niebezpiecznych stron.
    \item Ochrona przed atakami DDoS.
\end{itemize}
\subsection{Polityka zarządzania urządzeniami mobilnymi}
\begin{itemize}
    \item Wdrożenie MDM do zarządzania smartfonami.
    \item Szyfrowanie i zdalne usuwanie danych.
    \item Blokowanie instalacji nieautoryzowanych aplikacji.
\end{itemize}
    


\begin{thebibliography}{9}
\bibitem{Sekretariaty} {\it Organizacja i zakres działania sekretariatów sądowych}
  (\url{https://sip.lex.pl/akty-prawne/dzienniki-resortowe/organizacja-i-zakres-dzialania-sekretariatow-sadowych-oraz-innych-35642426}).
\bibitem{Endpoint} {\it Endpoint Manage Engine}
  (\url{https://www.manageengine.com/?pos=EndpointCentral&loc=MElogo}).
\bibitem{Nessus} {\it Tenable Nessus}
  (\url{https://www.tenable.com/}).
\bibitem{Nflo} {\it Opis polityki bezp. według Nflo}
  (\url{https://nflo.pl/slownik/polityka-bezpieczenstwa/#kluczowe-elementy-polityki-bezpieczenstwa}).
\bibitem {Endpoint Protector}{Endpoint Protector}
(\url{https://www.endpointprotector.com/}).
\end{thebibliography}



\end{document}
